\title{Bosonic Quantum Error Correction with a  Heavy \\ Fluxonium Control Qubit}
\author{Shoumik Chowdhury}
\prevdegrees{B.S., Yale University (2021)}
\department{Department of Electrical Engineering and Computer Science}
\degree{Master of Science}

% Degree Information 
\degreemonth{May}
\degreeyear{2024}
\thesisdate{May 17, 2024}

%% By default, the thesis will be copyrighted to MIT.  If you need to copyright
%% the thesis to yourself, just specify the `vi' documentclass option.  If for
%% some reason you want to exactly specify the copyright notice text, you can
%% use the \copyrightnoticetext command.  
%\copyrightnoticetext{\copyright IBM, 1990.  Do not open till Xmas.}

% If there is more than one supervisor, use the \supervisor command
% once for each.
\supervisor{William D. Oliver}{Henry Ellis Warren (1984) Professor of Electrical Engineering and Computer Science \& Professor of Physics}
\supervisor{Jeffrey A. Grover}{Research Scientist, Research Laboratory of Electronics}

% This is the department committee chairman, not the thesis committee
% chairman.  You should replace this with your Department's Committee
% Chairman.
\chairman{Leslie A. Kolodziejski}{Professor of Electrical Engineering and Computer Science \\ Chair, Department Committee on Graduate Students}

% Make the titlepage based on the above information.  If you need
% something special and can't use the standard form, you can specify
% the exact text of the titlepage yourself.  Put it in a titlepage
% environment and leave blank lines where you want vertical space.
% The spaces will be adjusted to fill the entire page.  The dotted
% lines for the signatures are made with the \signature command.
\maketitle

% The abstractpage environment sets up everything on the page except
% the text itself.  The title and other header material are put at the
% top of the page, and the supervisors are listed at the bottom.  A
% new page is begun both before and after.  Of course, an abstract may
% be more than one page itself.  If you need more control over the
% format of the page, you can use the abstract environment, which puts
% the word "Abstract" at the beginning and single spaces its text.

%% You can either \input (*not* \include) your abstract file, or you can put
%% the text of the abstract directly between the \begin{abstractpage} and
%% \end{abstractpage} commands.

% First copy: start a new page, and save the page number.
\cleardoublepage
% Uncomment the next line if you do NOT want a page number on your abstract and acknowledgments pages.
% \pagestyle{empty}

%%%%% New Blank Page %%%%%%
\newpage 
\hbox{}\par\vfill\centerline%
{THIS PAGE INTENTIONALLY LEFT BLANK}%
\vfill
\newpage
%%%%%%%%%%%

\setcounter{savepage}{\thepage}
\begin{abstractpage}
Bosonic codes store information in the phase space of a quantum harmonic oscillator and offer a hardware-efficient path towards quantum error correction (QEC), requiring only an oscillator and an auxiliary qubit for measurement and universal control. Of the many bosonic codes, the so-called Gottesman-Kitaev-Preskill (GKP) code stands out as one of the most robust to dominant physical decoherence mechanisms, but is severely limited by bit-flip errors in the control qubit. In this thesis, we develop a new approach for implementing GKP QEC in superconducting circuits based on using a heavy fluxonium as the auxiliary control qubit due to its inherent bit-flip protection. We demonstrate progress towards this in experiment by using a fluxonium in a 3D superconducting cavity architecture, and also propose novel strategies for moving future experiments to a fully 2D platform. 
\end{abstractpage}

% Additional copy: start a new page, and reset the page number.  This way,
% the second copy of the abstract is not counted as separate pages.
% Uncomment the next 6 lines if you need two copies of the abstract
% page.
% \setcounter{page}{\thesavepage}
% \begin{abstractpage}
% \input{abstract}
% \end{abstractpage}

%%%%% New Blank Page %%%%%%
\newpage 
\hbox{}\par\vfill\centerline%
{THIS PAGE INTENTIONALLY LEFT BLANK}%
\vfill
\newpage
%%%%%%%%%%%

\cleardoublepage