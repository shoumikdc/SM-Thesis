\appendix
\chapter{Appendix: Resonator Fitting\label{ch:AppA}}

A large part of the experimental work done in this thesis involves probing 3D cavity resonators using classical microwave fields. In doing so, we are able to infer various properties of the resonator such as its internal and coupling losses. In this appendix, we derive from first principles the resonance lineshape of a single-mode resonator coupled a readout transmission line. This simple example can be treated theoretically using the \textit{input-output formalism}, yet it is general enough to cover several of the experiments performed in this thesis. In order to bridge the gap between this model and our experimental setup, we also extend our analysis to the include (a) reflection measurements using a circulator, (b) asymmetric lineshapes due to so-called Fano resonances, and (c) the calculation of transmission coefficients when the mode is connected to separate transmission lines. 

\section{Input-Output Theory for Resonators}
\subsection{Single-Mode Reflection Measurements}

Let us consider a linear resonator mode $\hat{a}$ with angular frequency $\omega_0$ coupled to a transmission line. The Hamiltonian describing this mode is simply that of a quantum harmonic oscillator, $\hat{H} = \hbar\omega_0 \hat{a}^\dagger \hat{a}$. Without the transmission line, the dynamics of $\hat{a}$ are given in the Heisenberg picture by $\partial_t\hat{a} = i[\hat{H}, \hat{a}] = -i\omega_0\hat{a}$, which leads to oscillatory motion in phase space. However, if we include losses and coupling to the transmission line, the dynamics are modified in the form of a Heisenberg-Langevin equation \cite{gardiner1985input, walls1994quantum, gardiner2004quantum, clerk2010introduction}:

In order to derive the lineshape of a resonator subject to  

\subsection{}

\begin{equation}
    \odv*{\hat{a}}{t} = -i[\hat{H}, \hat{a}] + \cdots
\end{equation}

See Audrey Bienfait's thesis, Steven Touzard's thesis and quantumnetworks paper. 

\section{Experimental Considerations}

\subsection{Fitting the Background}
Low order polynomial fits etc.

\subsection{Fano Resonance: Asymmetry in the Lineshape}
Modify the fit function to 