\chapter{Conclusion\label{ch:6_Conclusion}}

Quantum error correction (QEC) is a vital component for the development of future fault-tolerant quantum computers and, as of this writing, represents the current north star that the field of quantum information science is working towards. Recent QEC experiments in 2023-2024 have demonstrated amazing progress towards building up register-based QEC codes, with many exciting avenues for progress being explored. Nevertheless, there is still much work left to be done to improve the state of quantum hardware and simultaneously find novel ways to reduce the resource overheard required for fault-tolerance. 

Within this context, bosonic QEC represents an elegant and hardware-efficient solution --- the idea here being to use bosonic qubits as a dynamically-protected base layer of QEC which we can concatenate into register-based codes. Bosonic qubits can achieve error rates far below the QEC threshold, and thus reduce the total number of qubits in the outer code. For superconducting circuits in particular, we believe bosonic codes will be a crucial part of future error correction schemes. Of the various codes that exist, the Gottesman-Kitaev-Preskill (GKP) bosonic code has recently been shown to be the most efficient at correcting realistic error channels and is thus a promising strategy for realizing a base layer bosonic qubit.  

The work in this thesis fits into the larger research landscape as a set of novel approaches for realizing GKP error correction with superconducting circuits using a heavy fluxonium as the auxiliary control qubit. Fluxonium has been shown to achieve $T_1$ lifetimes in excess of 1 ms, and serves as a bit-flip protected controller; this makes it a very attractive auxiliary qubit from the perspective of GKP error correction, since GKP codes are notably sensitive to control qubit bit flips (as we saw in simulation in Ch. \ref{ch:2_QEC}). Furthermore, fluxonium also offers the possibility of utilizing fast-flux control techniques in bosonic QEC. 

Our first approach to realizing bosonic QEC involved integrating a fluxonium qubit into a 3D superconducting post cavity architecture. We presented this experiment in Ch. \ref{ch:4_3DGKP} and successfully demonstrated our ability to (a) control the fluxonium, and (b) engineer a flux tunable dispersive shift to the storage cavity, which has been a long-standing experimental desideratum for bosonic qubits. While this portion of the experiment can be considered a success, we unfortunately later discovered that our specific implementation suffered from a flaw that caused the storage resonator lifetime to be significantly degraded, and was thus infeasible without a significant redesign of the 3D cavity and qubit. Despite this setback, we were ultimately able to take what we learned from the 3D project and apply it towards the design of two new experiments that are now currently underway. 

The first of these novel approaches is a re-implementation of our 3D design using a fully 2D architecture instead, with on-chip resonators in place of 3D cavities. We optimized this design in order to maximize the potential for QEC gain. Meanwhile, the second approach is a further development of this concept and introduces a coupler element that we expect will enable us to reach state-of-the-art GKP error correction performance. In the near term, we look forward to performing experiments with these devices and benchmarking their QEC capabilities. 

Looking ahead, both of these strategies offer promising avenues for future development, especially as 2D superconducting resonators continue to improve in quality. Our designs were formulated with extensibility in mind, and we could imagine eventually integrating either of them into a larger multi-logical-qubit array, each consisting of a GKP mode controlled by a heavy fluxonium qubit. We hope to touch on these ideas in due course. 