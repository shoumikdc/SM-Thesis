\chapter*{Acknowledgements}
\begin{onehalfspacing}

The past three years have been a period of immense growth for me, both scientifically and personally. I've learned \textit{so much} about what it means to be a scientist, and have gotten to experience the full spectrum of ups and downs that comes with it. Overall, though, it has been a joyful, enriching time in my life, and this feeling is largely thanks to the mentorship, support, kindness, and friendship of the wonderful people around me. Here, I would like to express my immense gratitude for these people.

I would like to first thank my advisor, Prof. William Oliver, for his guidance and vision. Will leads by example, and is an excellent role model in research and life. His ``no kluge'' approach to science and engineering is something I truly value, and his commitment to producing high-quality research is a constant source of motivation. One of Will's favorite phrases ``when you do things, things get done'' (i.e. always being proactive) is something I am actively trying to cultivate in myself going forward. Most of all, however, I appreciate Will's dedication to maintaining a warm, uplifting and kind lab culture, where people work hard and have fun too. This has made my time in the Engineering Quantum Systems (EQuS) group a joy. I'd be remiss not to also extend this thanks to the rest of EQuS leadership: Dr. Jeffrey Grover, Dr. Kyle Serniak, Dr. Simon Gustavsson, Dr. Joel Wang, Prof. Terry Orlando, and of course our administrative officer Chihiro Watanabe. I'd like to specifically thank Jeff for serving as my thesis mentor over the past several months and for his help in getting this thesis over the finish line, as well as for being a considerate and positive presence in our lab. I also want to thank Kyle for the many insightful discussions about device fabrication and design; I'm excited to now be working more closely with him on quasiparticles. Kyle is also the one that convinced me (way back) to join our lab band, \textit{The Harmonic Modes}, which has helped me reconnect with playing music in a way I didn't think I'd get to again. I look forward the band's next gig!

While PhDs can often be a bit of a solitary undertaking, I'm fortunate in that that hasn't been my experience \textit{at all}. Instead, I have had the distinct pleasure of working incredibly closely and collaboratively with Dr. Max Hays and Shantanu Jha. The 3 of us joined EQuS at the same time (all coming from Yale), and ended up forming the bosonic error correction subteam. Both Max and Shantanu are incredibly talented, and I'm often in awe of them. I think getting paired up with Max as a mentor is probably the single best thing that could have happened for my overall grad school trajectory, and I owe a lot of my development as a researcher these past three years to him. I admire Max's ability to go deep into any topic he puts his mind to, as well as his intuitive understanding of all aspects of physics (be it experimental design, abstract theory, numerical simulation, or measurement). He is a great friend and I'm glad he'll be staying on in EQuS. I'd also like to acknowledge Max's help in editing and finalizing this thesis. Now, Shantanu and I have known each other for the past 7 years. It's probably not too common to get to work on a PhD side-by-side with one of your closest friends and college roommates, but having done just that these past 3 years, I wouldn't have had it any other way. Shan is an excellent scientist and collaborator, and his work ethic is second to none; it's been a pleasure to work with and learn from him. I'm excited for us to keep developing new experiments, writing open-source software, and pushing ourselves for the rest of our PhDs and beyond. I'd encourage all readers of this thesis to also check out Shantanu's SM thesis, which covers many complementary aspects of our projects. 

My next round of thanks goes to all of the postdocs and graduate students in EQuS, past and present, from whom I have learned so much in these first few years of grad school. I can say without a doubt that I've enjoyed the company of every single person in the group throughout my time here, and I'm very grateful for the camaraderie and many memories in and outside of the lab. I thank Sameia Zaman in particular, who joined the group the same year as Shantanu and me, and who always checks in to make sure we are all keeping up with the various requirements. I would also like to give special thanks to to Dr. Miuko Tanaka for helping etch the first 3D cavities we measured; to Dr. Aranya Goswami for his help in the cleanroom and in optimizing our 3D cavity fab process; to Dr. Agustin Di Paolo for teaching me about all things related to Floquet theory; to Dr. Patrick Harrington for many lively discussions about measurements and readout, and for being a great fridge companion in Leiden 2; and of course, to Dr. R\'eouven Assouly to whom I owe a debt of gratitude. R\'eouven is a wizard in the lab and has probably helped debug every part of the experiment at some point or another: be it programming the OPX, wrangling with instrument IPs, messing around with the fridge, or ``jury rigging'' bobbins onto a 3D cavity. I definitely feel more comfortable around the lab after working with him. I'm also grateful for his help in identifying the design flaw in our 3D cavity package and his willingness to brainstorm creative ideas to fix it during an otherwise stressful couple of months in the 3D-GKP project where nothing seemed to work. At this point, I'd also like to thank the rest of the MIT Lincoln Laboratory team that helped fabricate most of the qubit devices used in this thesis: Michael Gingras, Jeff Knecht, Bethany Niedzielski, Jonilyn Yoder, Mollie E Schwartz, and Hannah Stickler.

Outside of research, I am grateful to the many friends I have made in Boston/Cambridge. I'll give a special mention to my friends on the Grad Rat ring committee. I'd also like to thank my former roommate Daud Shad, as well as my good friend Sakinah Master whose apartment (shared with Aashna) become my second home. 

I'd like to dedicate this thesis to my family members: my parents Sumit and Amrita, my younger sister Aishani, and the rest of my family in Mumbai --- Nana, Chetan, Shalini, Vardaan, and Shaurya. My parents are my biggest source of inspiration. They pushed me to work hard in life (really, just following their example) and have supported me unconditionally in all of my endeavors. Everything I am today is thanks to them. Aishani is a trusted confidante at home, and I'm glad to have grown up with her. My family has helped me believe I am capable of anything through their love and encouragement, and I always strive to make them proud.

And finally, I'd like to thank my best friend and life partner Aashna Shah. She has been there for me every step of the way, and brings light and joy to our life. I'm beyond grateful for her unconditional love and belief in me, and for making these past 3 years together in Boston (and the 9 years before that) so much fun. Aashna helps brighten even the roughest days, and has really taken care of me; I certainly wouldn't have been able to finish writing this thesis on time if not for her encouragement. I'm excited for us to now get our master's degrees together (she graduates from the Harvard School of Public Health next week), and for our continued adventures in life. 





\vfill
{\small 
\noindent \textbf{Funding acknowledgement:} This research was funded in part by the Amazon Web Services (AWS) Center for Quantum Computing and in part by the MIT Jacobs Presidential Fellowship. I would like to acknowledge support from the National Science Foundation Graduate Research Fellowship (GRFP) under Grant No. 1745302, as well as support from LPS/ARO under Grant No. W911NF-23-1-0045. Any opinions, findings, and conclusions or recommendations expressed in this thesis are my own and do not necessarily reflect the views of the National Science Foundation or LPS/ARO.
}

% .








\end{onehalfspacing}