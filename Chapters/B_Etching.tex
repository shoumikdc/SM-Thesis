\chapter{Appendix: Fabricating High-Q 3D Cavities \label{ch:AppB}}

In this brief appendix, we will review the standard operating procedures that we followed when fabricating aluminium 3D cavity resonators. Our overall procedure was developed through several iterations of cavities and so the version we present here is the final process that we eventually converged on. All of the 3D cavities used in this thesis were machined out of high-purity bulk aluminium with a grade of 5N5 (99.9995\% purity), as is typical for Al post cavity designs in past circuit QED experiments \cite{reagor2013reaching}. After designing and simulating the cavity, we submitted our specifications to the MIT Central Machine Shop for the actual machining of the device. Once we received the final machined cavities, the next steps were cleaning and etching; we performed both of these in the MIT.nano cleanroom.

For cleaning, we first dipped the cavity into a beaker of 1-methyl-2-pyrrolidone (NMP) for approximately a minute, until the color of the NMP changed to brown (removing any dirt or oils on the cavity left over from machining). We then did a more thorough cleaning by sequential sonication of the sample in (fresh) NMP, then acetone, and finally isopropyl alcohol (IPA). The solvents are chosen in the typical order of decreasing `strength' in order to remove machine oils and residues. We sonicate at a medium setting for 5 minutes each, transferring quickly between beakers to prevent solvent residues forming in air. After the IPA, we finally blow dry the sample with $N_2$ gas. 

Once the cavity is clean, we next proceed with the critical step of chemical etching in acid. The purpose of the etch is to treat surface imperfections and roughness of the aluminium, and it is a crucial ingredient for achieving high quality factor 3D resonators. We etch using the commonly-available Transene Aluminium Etchant Type A, which is a mix of nitric and phosphoric acid. The acid is heated to 50$^\circ$C using a hot plate and is temperature-controlled via a PID loop and thermometer. In our case, we placed the cavity in a teflon basket before lowering it into the acid bath along with a magnetic stir bar. We etched for 4 hours in total but replaced the acid every 45 minutes to prevent saturation. In each segment, we see the color of the acid change as the reaction proceeds as well as bubbles forming. We found it to be important to reorient the cavity within the acid in each segment to allow the bubbles to rise to the surface (and not get trapped within the cavity or below it). In our experience, this led to the best and most uniform etches, which indeed we saw translated into higher resonator lifetimes ---  after our most successful etch, we were able to measure cavities with single-photon lifetimes above 3 ms. We owe a special thanks to Miuko Tanaka for etching the first batch of 3D cavities, as well as to Aranya Goswami for his help in optimizing our etch process.

Following the acid treatment, we rinse the cavity in DI water for 5 minutes or more and then blow dry with $N_2$ gas. At this point, the cavity should have high lustre and the grain boundaries of the aluminium should be clearly visible. We finally perform another round of sonication in acetone and IPA for 3 minutes each before finally blow drying in $N_2$ again\footnote{We note that this process is slightly different to that in Ref. \cite{reagor2013reaching}, where the authors rinsed cavities in methanol following the acid etch. We did not try this yet, but leave it as an avenue for future 3D cavities we make.}.


\printbibliography[heading=subbibliography, title = References]